%\documentclass[referee]{aa}
\documentclass{aa}
\pdfoutput=1
\usepackage{natbib}
\usepackage{graphicx}
\usepackage{url}
\usepackage{txfonts}

\bibpunct{(}{)}{;}{a}{}{,} % to follow the A&A style

\usepackage{color}
\newcommand{\redtext}[1]{{\color{red} #1}}

\graphicspath{{./figs/}}

\begin{document}

\title{Machine Learning \& state definitions in Cyg X-1} 
\subtitle{XXX}

\titlerunning{XXX}

\author{\mbox{D.~Huppenkothen\inst{\ref{affil:nyu}}} \and
\mbox{V.~Grinberg\inst{\ref{affil:mit}}}
} 
\offprints{V.~Grinberg,\\ e-mail: {grinberg@space.mit.edu}}
\institute{ 
NYU\label{affil:nyu}
\and
Massachusetts Institute of Technology, Kavli Institute for
  Astrophysics and Space Research, Cambridge, MA 02139, USA
 \label{affil:mit}
}
\date {Received: --- / Accepted: ---}

\abstract{add something}

\keywords{stars: individual: Cyg~X-1 -- X-rays:
  binaries -- binaries: close}

\maketitle

\section{Intro}

States in black hole binaries: the idea. Connection to geometry. But
also note that the states are (mostly) a continuum.

A lot of open questions: Unclear origin of power law component (jet
vs. comptonization). Unclear origin of the lag.

Mention Belloni's thesis that the timing data are actually the
clearest markers of state transitions? Does the jet-line coincide with
changes in timing behavior? (Re-read Fender+ 2009)



Why machine learning method make so much sense. More unbiased then
humans. Can select what is actually more important without involving
wishful thinking (and egos).

XXX Stuff on machine learning XXX

Let's use machine learning. See what it tells us about what is
important. Let's compare it to the human made predictions and see
whether the different states can be recovered at all, i.e., whether
they exist.

Why Cyg is ideal for this kind of studies: bright, persistent = enough
observations, well understood behavior.

Cyg X-1 also a special case: HMXB, persistent, no hysterisis visible,
at least no clear q-track. Soft state with a hard tail, a lot of
variability in the hard tail. But similar in other regards, e.g. lag
(REF altamirano new paper?). Also, hints of two hard states (KP 2003?
Somewhere else?) XXX Be nice here and cite a lot of other people who
have worked with the data not to be biased: Shaposhnikov, Ibragimov,
Axelssson XXX

\section{Data \& Methods}

\subsection{Quick overview over RXTE data}

We use data from a bi-weekly observational campaign of Cyg X-1 with
RXTE performed between 1999 and 2011. We use Proportional Counter
Array \citep[PCA;][]{Jahoda_2006a} data.  Detailed description of data
selection and description of the calculation methods to obtain the
timing data are given in \citep{Grinberg_2014a}. All data reduction
performed with HEASOFT 6.11. All timing value calculations including
the deadtime corrections were done following the \citet{Nowak_1999a}
and \citet{Pottschmidt_2003b} using ISIS~1.6.2
\citep{Houck_Denicola_2000a,Houck_2002,Noble_Nowak_2008a}.

Give a short overview of what can be done with
the data given the limited data modes at all.

\begin{table}
\begin{tabular}{l|cc}
band &energy [keV]& PCA \texttt{std2}
  channels\\
\hline
band 1 &$\sim$2.1--4.5\,keV& 0--10 \\
band 2 &$\sim$4.5--5.7\,keV& 11--13 \\
band 3 & $\sim$5.7--9.4\,keV& 14--12 \\
band 4 &$\sim$9.4--15\,keV& 23--35 \\
\end{tabular}
\end{table}


We first go for a set of data that are as model-independent as
possible and that means that it's mainly timing data. In particular we
use:

\begin{itemize}
\item the average total countrate in 2.1--15\,keV (PCA \texttt{std2}
  channel 0--35) which can be seen as a proxy for the source
  intensity.
\item the average total countrates in $\sim$2.1--4.5\,keV (channels 0--10),
  $\sim$4.5--5.7\,keV (channels 11--13), $\sim$5.7--9.4\,keV (channels
  14--22), and $\sim$9.4--15\,keV (channels 23--35) energy bands.
\item fractional rms in the 0.125-–256\,Hz range calculated in
  $\sim$2.1--4.5\,keV (channels 0--10),
  $\sim$4.5--5.7\,keV (channels 11--13), $\sim$5.7--9.4\,keV (channels
  14--22), and $\sim$9.4--15\,keV (channels 23--35) energy bands.
\end{itemize}

\subsection{Quick overview over machine learning methods used}

\section{What we did \& Results without yet interpreting them}

What does machine learning actually tell us about states.

\section{Discussion}

\begin{itemize}

\item Which parameters do actually drive the machine defined state
  definitions?

\item So how does the battle man vs. machine fare?

\item Tie it back to physics?

\end{itemize}

\section{Conclusions}

\begin{acknowledgements}
  Support for this work was provided by NASA through the Smithsonian
  Astrophysical Observatory (SAO) contract SV3-73016 to MIT for
  Support of the Chandra X-Ray Center (CXC) and Science Instruments;
  CXC is operated by SAO for and on behalf of NASA under contract
  NAS8-03060.
\end{acknowledgements}

\bibliographystyle{aa} 
\bibliography{mnemonic,aa_abbrv,references}

\end{document}
