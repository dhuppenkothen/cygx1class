%\documentclass[referee]{aa}
\documentclass{aa}
\pdfoutput=1
\usepackage{natbib}
\usepackage{graphicx}
\usepackage{url}
\usepackage{txfonts}

\bibpunct{(}{)}{;}{a}{}{,} % to follow the A&A style

\usepackage{color}
\newcommand{\redtext}[1]{{\color{red} #1}}

\graphicspath{{./figs/}}

\begin{document}

\title{Machine Learning \& state definitions in Cyg X-1} 
\subtitle{XXX}

\titlerunning{XXX}

\author{\mbox{D.~Huppenkothen\inst{\ref{affil:nyu}}} \and
\mbox{V.~Grinberg\inst{\ref{affil:mit}}}
} 
\offprints{V.~Grinberg,\\ e-mail: {grinberg@space.mit.edu}}
\institute{ 
NYU\label{affil:nyu}
\and
Massachusetts Institute of Technology, Kavli Institute for
  Astrophysics and Space Research, Cambridge, MA 02139, USA
 \label{affil:mit}
}
\date {Received: --- / Accepted: ---}

\abstract{add something}

\keywords{stars: individual: Cyg~X-1 -- X-rays:
  binaries -- binaries: close}

\maketitle

\section{Intro}

Black hole binaries (BHBs) show distinct emission regimes, or
``states'' in the X-rays that are connected to different behavior in
other bands and thought to correspond to different configarations of
accretion and ejection flows such as accretion disk, jets, accretion
disk corona, and disk winds. 


The states have originally been defined based on spectral features:
in the hard state, the X-ray spectrum above $\sim$2\,keV is dominated
by power law emission where the photon spectrum is proportional to
$E^{-\Gamma}$ with a photon index $\Gamma \sim 1.7$. In the
soft state thermal emission from the accretion disk is prominent and
the contribution from a steeper power law lower
\citep[e.g.,][]{McClintock_Remillard_2006a_book}.  The extremes are
connected by intermediate states \citep{Fender_2009a,Belloni_2010a}.

Further features emerged: radio. Timing.

The empirical evolution of a BHB is usually presented in terms of
hardness-intensity diagrams (HIDs). Transients BHBs in outburst trace
a \textsf{q}-shaped track \citep{Homan_2005a,Fender_2004a}; the tracks of persistent sources can be less
clear. Neutron stars, CVs, and AGNs track similar patterns, so that
understanding states is paramount to understanding the complex
accretion-ejection interplay in all kinds of objects.


States in black hole binaries: the idea. Connection to geometry. But
also note that the states are (mostly) a continuum.

A lot of open questions: Unclear origin of power law component (jet
vs. comptonization). Unclear origin of the lag.

Mention Belloni's thesis that the timing data are actually the
clearest markers of state transitions? Does the jet-line coincide with
changes in timing behavior? (Re-read Fender+ 2009)



Why machine learning method make so much sense. More unbiased then
humans. Can select what is actually more important without involving
wishful thinking (and egos).

XXX Stuff on machine learning XXX

Let's use machine learning. See what it tells us about what is
important. Let's compare it to the human made predictions and see
whether the different states can be recovered at all, i.e., whether
they exist.

Why Cyg is ideal for this kind of studies: bright, persistent = enough
observations, well understood behavior.

Cyg X-1 also a special case: HMXB, persistent, no hysterisis visible,
at least no clear q-track. Soft state with a hard tail, a lot of
variability in the hard tail. But similar in other regards, e.g. lag
(REF altamirano new paper?). Also, hints of two hard states (KP 2003?
Somewhere else?) XXX Be nice here and cite a lot of other people who
have worked with the data not to be biased: Shaposhnikov, Ibragimov,
Axelssson XXX

\section{Data \& Methods}

\subsection{Quick overview over RXTE data}

We use data of Cyg X-1 taken after MJD~51259, i.e., starting when
calibration epoch~4, with the Proportional Counter Array
\citep[PCA;][]{Jahoda_2006a} onboard Rossi X-ray Timing Explorer
(RXTE). The transition between calibration epoch~3 and~4 is marked by
an abrupt change in channel energies \citep[see e.g.][]{Garcia_2014a}
so that we would not be able to use consistent channel energies for
the calculation of timing parameters across epoch boundaries since the
observations are taken in binned mode. We found out that this matters
for our data visually inspecting them with a
t-SNE\citep{van_der_Maaten_2008a} implementation in XXX details XXX.

Detailed description of data selection and description of the
calculation methods to obtain the timing data are given in
\citep{Grinberg_2014a}. All data reduction performed with HEASOFT
6.11. All timing value calculations including the deadtime corrections
were done following the \citet{Nowak_1999a} and
\citet{Pottschmidt_2003b} using ISIS~1.6.2
\citep{Houck_Denicola_2000a,Houck_2002,Noble_Nowak_2008a}.

\begin{table}
\caption{Energy bands, their energies and PCA \texttt{std2}
  channel boundaries used in this work.}\label{tab:channels}
\begin{tabular}{lcc}
\hline\hline
band &energy [keV]& PCA \texttt{std2}
  channels\\
\hline
band 1 &$\sim$2.1--4.5\,keV& 0--10 \\
band 2 &$\sim$4.5--5.7\,keV& 11--13 \\
band 3 & $\sim$5.7--9.4\,keV& 14--12 \\
band 4 &$\sim$9.4--15\,keV& 23--35 \\
total band & $\sim$2.1--15\,keV & 0--35 \\
\hline
\end{tabular}
\end{table}


We first go for a set of data that are as model-independent as
possible and that means that it's mainly timing data. In particular we
use (see Table~\ref{tab:channels} for the energy band definitions):

\begin{itemize}
\item the average countrate in the total energy band which can be seen
  as a proxy for the source intensity.
\item the average countrates in bands 1, 2, 3, and 4.
\item fractional rms in the 0.125-–256\,Hz range calculated in
  band 1, 2, 3, and 4.
\item fractional rms in energy band 1 calculates in the frequency
  ranges 0.25--2.0\,Hz and 2.0--16\,Hz bands (this corresponds to two
  of the frequency ranges used by \citet{Heil_2015a}, we cannot use
  their other two frequency ranges because we calculate our PSDs and
  the higher order statistics of lags and coherence on shorter
  segments, especially to be able to do the higher order
  statistics). 16\,Hz is also below the buffer overflow problems we
  get for the bright observations (REF).
\item the average timelag between band 4 and band 1 in the 3.2--10\,Hz
  range which is the typical value for the timelag used in the
  literature that is known to show changes of behavior with
  states. Because of the overall $\sim f^{-0.7}$ shape of the timelage
  we have to calculate the lag in a rather thin frequency range.
\item the average value of the coherence function between band 4 and
  band 1 in the 0.125--5.0\,Hz range.
\end{itemize}

\subsection{Quick overview over machine learning methods used}

\section{What we did \& Results without yet interpreting them}

What does machine learning actually tell us about states.

\section{Discussion}

\begin{itemize}

\item Which parameters do actually drive the machine defined state
  definitions?

\item So how does the battle man vs. machine fare? Is machine better
  at recognizing states?

\item do we actually see three states or just two of them?

\item is the hard state really one state or are there two distinct
  behavior pattern?

\item Tie it back to physics?

\end{itemize}

\section{Conclusions}

This is why we need more machine learning approaches.

This is what new stuff we learned about the source.

\begin{acknowledgements}
  Support for this work was provided by NASA through the Smithsonian
  Astrophysical Observatory (SAO) contract SV3-73016 to MIT for
  Support of the Chandra X-Ray Center (CXC) and Science Instruments;
  CXC is operated by SAO for and on behalf of NASA under contract
  NAS8-03060.
\end{acknowledgements}

\bibliographystyle{aa} 
\bibliography{mnemonic,aa_abbrv,references}

\end{document}
